
\documentclass[11pt]{article}
\usepackage{url,charter,graphicx}% http://ctan.org/pkg/{url,charter,graphicx}
\usepackage[margin=.8in, bottom=.5in, top=.5in]{geometry}% http://ctan.org/pkg/geometry
\setlength{\parindent}{0pt}
\setlength{\parskip}{10pt}
\pagestyle{empty}
\begin{document}
	\hspace{2.8in}
	\begin{minipage}{2in}
	\includegraphics[width=100pt]{ubc}
	\end{minipage}
\hfill
	\begin{minipage}{3in}
	\today \\[\jot]
	Harold N. Eyster\\
	%429-2202 Main Mall \\
	%Vancouver, BC V6T 1Z4 \\
	%Phone: (734) 719-0359 \\
	haroldeyster@gmail.com
	\end{minipage}
	\medskip
	\begin{tabular}{@{}l}
		Melinda D. Smith\\
		Editor-in-Chief \\
		\textit{Climate Change Ecology} \\
	\end{tabular}
	
	\bigskip
	
	Dear Dr. Smith,

	\medskip
	Please consider our revised manuscript entitled, ``Comparisons in the native and introduced ranges reveal little evidence of climatic adaptation in germination traits" for publication as an Original Research Paper in \textit{Climate Change Ecology}. \par
	Plant invasions profoundly transform natural communities and can also serve as natural experiments to investigate how plants may react to climate change, where the climate change experienced when a plant colonizes a new environment is a proxy for the anthropogenic climate change that plants are experiencing now. However, the underlying mechanisms that underlie plant invasions and responses to climate change remain disputed. While some studies suggest rapid evolution after reaching a new habitat determines invasion success, other studies suggest habitat generalists can immediately flourish in a range of habitats.\par 
	%We address this controversy by leveraging the power of three key study design features: 1) native and nonnative populations of the same species planted under multivariate environmental conditions, 2) multilevel Bayesian modeling, and 3) multiple species. First, we used growth chambers to test if populations from species' native versus non-native ranges responded differently to multivariate environmental cues---equivalent to winter crossed with spring temperature regimes. Second, we used multilevel Bayesian modeling to control for local population, parent plant, and species, thus integrating over these complex factors in a united model. Our experimental design combined with our modeling approach enabled us to study seven highly invasive species (both dicots and monocots) at once, thereby providing more general estimates. Multi-species studies, such as ours, are critical to understanding the generality of findings, but still rare today given design hurdles; we suggest our approach may be a useful template for future studies.  \par % We believe combining similar growth chamber designs with Bayesian modeling approaches, which integrate across multiple levels of variance (species, population, seed family), provides a tractable approach for other populations, other traits, and other combinations of climate factors (including precipitation). 
	Our results show that post-invasion rapid evolution of germination and growth traits is unlikely to be essential for invasion success. Instead, broad environmental tolerance can be key. Furthermore, current estimates of invaders' responses to climate may be useful for forecasting responses to future climate change. However, we did find limited evidence that our study species have adapted to shorter winters $\times$ warmer springs. This suggests that plants may evolve in response to specific seasonal climate regimes that are not commonly tested today, but may be important for range expansions under climate change. \par 
	Constructive and thorough comments from two reviewers have helped us improve our manuscript. We have re-run all our models (with more intuitive reference levels),  overhauled most of the figures and the Discussion, and made many other improvements throughout. We have added two new figures: one to aid understanding of our sampling, experimental, and statistical design, and one to aid understanding of our key interaction results, which were formerly not presented intuitively. We have also modified three figures in the main text, four figures in the Supplement, and six tables in the Supplement. The figures and the main text now reflect more consistent terminology around temperature treatments and seed origin. Our Discussion has been re-written; it now makes the limited sample size in the introduced range more explicit (as does the updated Methods), and provides a more organized and coherent discussion of the interpretations and limitations of our study. We feel the new submission is much improved and detail our changes in the attached pages. Both authors substantially contributed to this work and approved of this version for submission.  

We hope that you will find it suitable for publication in \textit{Climate Change Ecology} and look forward to hearing from you. \\  %A. Ettinger, C. Chamberlain, and D. Buonaito have previously reviewed the manuscript. We recommend the following reviewers: XXXX. \par 
	\null\hfill
	\begin{tabular}{l@{}}
		Sincerely, \\[1\normalbaselineskip] %changed from 5
		\includegraphics[width=100pt]{sig}\\
		Harold N.\ Eyster
	\end{tabular}
	
\end{document}

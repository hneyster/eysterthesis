
\documentclass{article}
\usepackage{url,charter,graphicx}% http://ctan.org/pkg/{url,charter,graphicx}
\usepackage[margin=1in]{geometry}% http://ctan.org/pkg/geometry
\setlength{\parindent}{0pt}
\setlength{\parskip}{12pt}
\pagestyle{empty}
\begin{document}
	
	%	\includegraphics[width=100pt]{example-image-a} \par
	%	\rule{\textwidth}{1.5pt}
	
	\null\hfill
	\begin{tabular}{l@{}}
		\today \\[\jot]
		Harold N. Eyster\\
		429-2202 Main Mall \\
		Vancouver, BC V6T 1Z4 \\
		Phone: (734) 719-0359 \\
		Email: haroldeyster@gmail.com
	\end{tabular}  
	
	\medskip
	
	\begin{tabular}{@{}l}
		David J. Gibson \\
		Executive Editor \\
		\textit{Journal of Ecology} \\
	\end{tabular}
	
	\bigskip
	
	Dear Dr. Gibson,
	
	\medskip
	Please consider our paper, entitled “Invader success and changing climate: Comparisons in the native and introduced range of seven plant species” for
publication as a “Research Article” in \textit{Journal of Ecology}. \par
	Plant invasions profoundly transform natural communities, agricultural systems, and ecosystem services. The magnitude of these effects has motivated a prolonged interest in understanding the underlying mechanisms that enable some plants to invade. Furthermore, plant invasions can serve as natural experiments to address theoretical questions in ecology and evolution.  \par 
	Yet, despite substantial research effort, these underlying mechanisms remain disputed. Some studies suggest that whether or not a species rapidly evolves upon reaching a new habitat determines its invasive success. Other studies suggest that plants need not evolve; instead, habitat generalists can immediately flourish in a range of habitats upon reaching them.\par 
	Our study addresses this controversy by leveraging the power of three key features: 1) native and nonnative populations of the same species planted together in growth chambers under various environmental conditions, 2) multilevel Bayesian modeling, and 3) multiple species. First, the growth chambers enabled us to test how different populations responded to a range of environmental cues, thereby exposing whether the populations have diverged since colonization. Second, the multilevel Bayesian modeling enabled us to control for local population, what individual plant each seed came from (seed family), and species, which can normally cloud interpretations. And finally, our experiments included seven highly invasive species (both dicots and monocots), thereby giving us some latitude to generalize our conclusions. \par 
	By marshaling these three key features, our study suggests that broad environmental tolerance, rather than rapid evolution, gives plants the capacity to invade novel environments. Thus, to predict what plants may pose an invasion risk, managers should look towards those plants that are already weedy and widespread in their native ranges. However, we did find limited evidence that our species have adapted to shorter winters and longer summers, as expected amidst anthropogenic climate change. This indicates that invasive plants may be able to adapt to climate change. 
	
	These results settle a longstanding theoretical controversy in ecology, and suggests a profitable experimental design for future research. Given the theoretical and methodological significance of this study, we contend that this work is well-suited for publication in the high-visibility forum that \textit{Journal of Ecology} provides. \par 
A. Ettinger, C. Chamberlain, and D. Buonaito have previously reviewed the manuscript. We recommend the
following reviewers: XXXX. 
	\bigskip
	
	\null\hfill
	\begin{tabular}{l@{}}
		Sincerely, \\[5\normalbaselineskip]
		Harold N.\ Eyster
	\end{tabular}
	
\end{document}


\documentclass{article}
\usepackage{url,charter,graphicx}% http://ctan.org/pkg/{url,charter,graphicx}
\usepackage[margin=1in]{geometry}% http://ctan.org/pkg/geometry
\setlength{\parindent}{0pt}
\setlength{\parskip}{12pt}
\pagestyle{empty}
\begin{document}
	
	%	\includegraphics[width=100pt]{example-image-a} \par
	%	\rule{\textwidth}{1.5pt}
	
	\null\hfill
	\begin{tabular}{l@{}}
		\today \\[\jot]
		Harold N. Eyster\\
		429-2202 Main Mall \\
		Vancouver, BC V6T 1Z4 \\
		Phone: (734) 719-0359 \\
		Email: haroldeyster@gmail.com
	\end{tabular}  
	
	\medskip
	
	\begin{tabular}{@{}l}
		David J. Gibson \\
		Executive Editor \\
		\textit{Journal of Ecology} \\
	\end{tabular}
	
	\bigskip
	
	Dear Dr. Gibson,
	
	\medskip
	Please consider our paper, entitled “Invader success and changing climate: Comparisons in the native and introduced range of seven plant species” fo publication as a “Research Article” in \textit{Journal of Ecology}. \par
	Plant invasions profoundly transform natural communities, agricultural systems, and ecosystem services. The magnitude of these effects has led to interest in understanding the underlying mechanisms that enable some plants to invade. Furthermore, plant invasions can serve as natural experiments to address theoretical questions in ecology and evolution.  Despite substantial research effort, the underlying mechanisms for plant invasions remain disputed. While some studies suggest rapid evolution after reaching a new habitat determines invasion success, other studies suggest habitat generalists can immediately flourish in a range of habitats.\par 
	We address this controversy by leveraging the power of three combined features of our study design: 1) native and nonnative populations of the same species planted together under multivariate environmental conditions, 2) multilevel Bayesian modeling, and 3) multiple species. First, we used growth chambers to test if populations from species' native versus non-native ranges responded differently to multivariate environmental cues (equivalent to winter crossed with spring temperature regimes). Second, we used multilevel Bayesian modeling to control for local population, seed family (what individual plant each seed came from), and species in one model, thus integrating over these complex factors in one united model. Our experimental design combined with our modeling approach allowed us to study seven highly invasive species (both dicots and monocots) at once, thereby providing estimates of effects across species. Multi-species study, such as ours, are critical to understanding the generality of findings but still rare today given design hurdles; we suggest our approach may be a useful template for other studies.  \par % We believe combining similar growth chamber designs with Bayesian modeling approaches, which integrate across multiple levels of variance (species, population, seed family), provides a tractable approach for other populations, other traits, and other combinations of climate factors (including precipitation). 
	Our results show that post-introduction rapid evolution of germination and growth traits is unlikely to be essential for all invasion success. Instead, it seems that broad environmental tolerance is key to invasion success for these seven species. Thus, to predict what plants may pose an invasion risk, managers should look towards those species that are already weedy and widespread in their native ranges. However, we did find limited evidence that our study species have adapted to shorter winters x longer summers. This suggests that plants may evolve in reponse to specific seasonal climate regimes that are not commonly tested today, but may be important to invasions and to range expansions with climate change. 
%EMWAug19: I wouldn't say 'settle' -- a little too strong I think. But to shorten I also suggest cutting this ...
We believe this work is well-suited for publication in the high-visibility forum that \textit{Journal of Ecology} provides, and hope you will agree.  A. Ettinger, C. Chamberlain, and D. Buonaito have previously reviewed the manuscript. We recommend the following reviewers: XXXX. \par 
	\bigskip
	
	\null\hfill
	\begin{tabular}{l@{}}
		Sincerely, \\[5\normalbaselineskip]
		Harold N.\ Eyster
	\end{tabular}
	
\end{document}
